\documentclass[a4paper,11pt]{article}
\usepackage[T1]{fontenc}
\usepackage[utf8]{inputenc}
\usepackage{lmodern}
\usepackage[francais]{babel}
\usepackage{setspace}
\usepackage{amsmath}
\usepackage{amssymb}
%\setlength{\bottummargin}{0pt}
\setlength{\topmargin}{0pt}
\title{Compte rendu de la réunion 14/01/2015}
\author{}

\begin{document}
\maketitle






\part*{Présent}
	
		\begin{itemize}
			
			\item Pascal GEORGI : Encadrant du projet
			\item Adil EL-QLAI
			\item Samuel FOBIS
			\item Amazigh HADDADOU
			
		\end{itemize}
		
\section{Ordre Du Jour}
		\begin{itemize}
		\begin{doublespacing}
			\item Bien Comprendre l'objectif du projet (de l'application à concevoir).
			\item Décomposition en étapes l'ensemble du projet.
			\item Définition, avec l'encadrant, les différentes fonctionnalités de l'application. 
			\item Le choix du langage et sur quelle plateforme mobile sera développée.
		\end{doublespacing}	
		\end{itemize}
		


\section{Informations échangées}

\paragraph{}
\begin{doublespacing}
		On a pu divisé le projet en deux parties de développement, la première a pour objectif d'avoir une présentation exploitable d'une matrice prise en photo et la deuxième consiste à appliquer les différents calculs mathématique.
	
		\paragraph{Première Partie :}
	
		
		\begin{itemize}
		
		\item Prendre en photo une matrice écrite sur tableau, sur une feuille au autre support.
		\item Utilisation de OCR (optical character recognition) la reconnaissance optique des caractères pour extraire de l'image toutes les données importante (crochets "[ ]" qui délimitent la matrice ou parenthèses, les trois point "..." dans le cas d'une grosse matrice, valeurs...)
		
		\item Avoir une représentation exploitable de la matrice grâce à OCR selon la forme de la matrice (simple ou les colonnes de la matrice sont une suite arithmétique ou géométrique )
		\item Affichage sur écran de la matrice extraite, possibilité d'utiliser une représentation XML pour affichage.
		
		\end{itemize}
		
	\end{doublespacing}	
		
		\paragraph{Deuxième Partie :}
		
		\begin{doublespacing}	
La mise en œuvre des différents algorithmes pour le calculs mathématique et tout les calculs se feront en fonction de l'ensemble dont lequel la matrice est défini (entiers, rationnels, corps fini...)			
			\begin{itemize}
			\item Le déterminant 
			\item Le Rang
			\item Inverser la matrice 
			\item Résolution d'un système linéaire  
					\begin{math}
					\begin{bmatrix}
					1 & 2 & 3 \\
					4 & 6 & 8 \\
					8 & 1 & 7 
					\end{bmatrix}
					\begin{pmatrix}
					x \\
					y \\
					z
					\end{pmatrix}
					= 
					\begin{pmatrix}
					2 \\ 3 \\ 4
					\end{pmatrix}
					\end{math}
			\end{itemize}

\paragraph{}			
 En fonction de temps, un autre objectif est d'intégrer une API pour optimiser les calculs sur des grosses matrice (exemple M):
 
\end{doublespacing}	

	\begin{center}
		
\begin{math}
M=
\begin{bmatrix}

   a & u_{0} & u_{1} & \ldots & u_n \\
   v_{0}  \\
   v_{1} \\
   \vdots \\
   v_{n}
   
   
\end{bmatrix}
\end{math}
\end{center}	

\part*{ La date de la Prochaine réunion }
\paragraph{} Une réunion est fixée pour la fin du mois de Janvier.



\end{document}
